\documentclass[a4paper]{article}
\usepackage{graphicx}
\renewcommand*\contentsname{Indice}
\begin{document}
\begin{titlepage}
    \centering
    \includegraphics{img/universita-degli-studi-di-padova.eps}
    {\huge\bfseries Realizzazione di Sans, un simulatore di sensori\par}
    \vspace{0.5cm}
	{\scshape\Large Corso di programmazione a oggetti\par}
	\vspace{1.5cm}
    {\raggedright \textsc{Studente} \\ \Large\itshape Lazzarin Tommaso\par}
    \vspace{0.1cm}
    {\raggedright\textsc{2075529}\par}
    \vfill
	{\large \textsc{Anno Accademico 2023-2024}}
\end{titlepage}
\clearpage
\tableofcontents
\clearpage
\section{Introduzione}
\subsection{Sintesi del programma}
Sans è un programma scritto in C++ che implementa il framework Qt per presentare le informazioni all'utente attraverso un'interfaccia grafica. Il suo compito è gestire, simulare, importare ed esportare sensori.
Il programma ne mette a disposizione tre tipi: quelli che rilevano un valore e la sua variazione nel tempo, quelli che rilevano una quantità e quelli che rilevano un evento.
Questa scelta è stata fatta in modo tale da utilizzare due tipi di grafici e una barra del progresso: 
\begin{itemize}
    \item Un grafico lineare per rappresentare i valori nel tempo;
    \item Un grafico a barre che rappresenta il numero di occorrenze di un evento in un periodo;
    \item Una barra del progresso che rappresenta una percentuale.
\end{itemize}
La selezione di questo tipo di sensori permette di avere una gerarchia di classi in cui le sottoclassi riutilizza le strutture dati già messe a disposizione dalla sua superclasse in maniera efficace.
Un altro punto fondamentale di questo progetto è il supporto alla persistenza dei dati, questo è ottenuto attraverso l'uso di JSON, un formato strutturato.
\subsection{Funzioni implementate}
Per la gestione dei singoli sensori:
\begin{itemize}
    \item Creazione
    \item Modifica
    \item Simulazione
    \item Cancellazione
    \item Importazione ed esportazione attraverso un file JSON
\end{itemize}
Per la gestione dell'insieme di sensori:
\begin{itemize}
    \item Barra di ricerca
    \item Importazione ed esportazione di un intero insieme di sensori attraverso un file JSON
    \item Ordinamento secondo la data di aggiunta o l'ordine alfabetico
\end{itemize}
\clearpage
\section{Implementazione}

\end{document}