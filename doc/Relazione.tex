\documentclass[a4paper]{article}
\usepackage{graphicx}
\renewcommand*\contentsname{Indice}
\begin{document}
\begin{titlepage}
    \centering
    \includegraphics{img/universita-degli-studi-di-padova.eps}
    {\huge\bfseries Realizzazione di un simulatore di sensori\par}
    \vspace{0.5cm}
	{\scshape\Large Corso di programmazione a oggetti\par}
	\vspace{1.5cm}
    {\raggedright \textsc{Studente} \\ \Large\itshape Lazzarin Tommaso\par}
    \vspace{0.1cm}
    {\raggedright\textsc{2075529}\par}
    \vfill
	{\large \textsc{Anno Accademico 2023-2024}}
\end{titlepage}
\clearpage
\tableofcontents
\clearpage
\section{Introduzione}
\subsection{Sintesi del programma}
Sensori è un programma scritto in C++ e implementa il framework Qt per presentare le informazioni all'utente attraverso un'interfaccia grafica. Il suo compito è gestire, simulare, importare ed esportare sensori.
Il programma ne mette a disposizione tre tipi: quelli che rilevano un valore e la sua variazione nel tempo, quelli che rilevano una quantità e quelli che rilevano gli eventi.
Questa scelta è stata fatta in modo tale da utilizzare vari tipi di grafici: 
\begin{itemize}
    \item Un grafico lineare per rappresentare i valori nel tempo;
    \item Un grafico a barre che rappresenta il numero di occorrenze di un evento in un periodo;
    \item Una barra del progresso che rappresenta un solo valore rispetto ad un limite.
\end{itemize}
L'interfaccia è composta da due componenti principali, il \textbf{browser} che ha il compito di rappresentare tutti i sensori attualmente disponibili e l'\textbf{inspector} che si occupa della gestione e della simulazione del sensore selezionato.
Questo progetto offre il supporto persistenza dei dati tramite l'uso di un formato strutturato chiamato JSON.
\section{Descrizione del modello}
\subsection{Polimorfismo}
\subsection{Persistenza dei dati}
\section{Funzionalità implementate (DA FINIRE PRENDENDO COME ESEMPIO QUELLA DEL MODELLO DI RELAZIONE)}
Per la gestione di un singolo sensore ho implementato le seguenti funzionalità:
\begin{itemize}
    \item Creazione;
    \item Rinominazione;
    \item Simulazione;
    \item Cancellazione.
\end{itemize}
Per la gestione dell'insieme di sensori:
\begin{itemize}
    \item Barra di ricerca per filtrare i sensori attraverso il nome o l'ID;
    \item Importazione ed esportazione di un intero insieme di sensori attraverso un file JSON.
\end{itemize}
\section{Rendicontazione ore}

\end{document}